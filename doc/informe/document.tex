\documentclass[12pt]{article}

%Packages
\usepackage[latin1]{inputenc}
% Esto es para que el LaTeX sepa que el texto está en español:
\usepackage[spanish]{babel}
\usepackage[x11names,table]{xcolor}

% Paquetes de la AMS:
%\usepackage[total={6in,11in},top=0.50in, left=1in]{geometry}
\usepackage[top=1in, left=1in, right=1in, bottom=1in]{geometry}
\usepackage{amsmath, amsthm, amsfonts}
\usepackage{graphics}
\usepackage{listings}
\usepackage{float}
\usepackage{epsfig}
\usepackage{amssymb}


\lstset{ %
	language=prolog, % lenguaje
	basicstyle=\normalsize\ttfamily,
	keywordstyle=\color{blue},
	commentstyle=\color{blue!50},
	backgroundcolor=\color{gray!9},
	identifierstyle = \color{gray!161},
	stringstyle = \color{yellow},
	numberstyle = \color{green},
	columns=fullflexible,
	showspaces=false
}



\newtheorem{thm}{Teorema}[section]
\newtheorem{cor}[thm]{Corolario}
\newtheorem{lem}[thm]{Lema}
\newtheorem{prop}[thm]{Proposición}
\theoremstyle{definition}
\newtheorem{defn}[thm]{Definicion}
\theoremstyle{remark}
\newtheorem{rem}[thm]{Observación}

\def\RR{\mathbb{R}}

\renewcommand{\labelenumi}{$\bullet$}
\newtheorem{definition}{Definici�n}[section]
\newtheorem{theorem}{Teorema}[section]
\newtheorem{corollary}{Corolario}[section]
\newtheorem{lemma}{Lema}[section]
\newtheorem{proposition}{Proposici�n}[section]
\newcommand{\statement}[3]{
	\begin{center}
		{ \fcolorbox {gray!11}{gray!11}{
				\begin{minipage}[h!]{\textwidth}
					\begin{#1}\label{#3}
						#2
					\end{#1}
				\end{minipage} } }
			\end{center}}
			\renewcommand{\proof}[1]{{\it Demostraci�n}\\ #1 \hfill\blacksquare}
\newcommand{\pagediv}[4]
{
	\begin{figure}[!h]
		\begin{minipage}[b]{#1\textwidth}
			#3			
		\end{minipage} \hfill 
		\begin{minipage}[b]{#2\textwidth}
			#4
		\end{minipage}
	\end{figure}
	
}



%define title
\author{
	Dalianys P\'erez Perera\\
	Dayany Alfaro Gonz�lez\\
	C-411 \\
}
\title{Azul   \\
	Programaci�n Declarativa
	}

\date{}
\begin{document}
%generates the title
\maketitle

\selectlanguage{spanish}

\newpage
%insert table contents
\tableofcontents
\newpage

\section {Estrategia}
Cada jugador tiene que decidir qu� azulejos va a tomar, de d�nde los va a tomar y d�nde los va a colocar, para esto se implement� una estrategia de selecci�n que se va describir a continuaci�n.
Como punto de partida es necesario tener un conjunto de todos los movimientos factibles a realizar, donde por movimiento factible se entiende que sea v�lido colocar en la fila seleccionada (o el suelo) los azulejos de un determinado color tomados de una de las f�bricas o el centro de la mesa. Para obtener esto se crearon todas las combinaciones de movimientos posibles, o sea, tanto v�lidos como inv�lidos y se filtraron mediante la evaluaci�n del predicado \lstinline|factible_move| para determinar si eran factibles o no.

Sea \lstinline|All_moves| el conjunto de movimientos factibles para el jugador \lstinline|ID|, va a estar representado por una lista de tuplas con el formato \lstinline|[(Source, Color, Amount, Stair, Chip),...]| que indica que se toman \lstinline|Amount| fichas de color \lstinline|Color|(1,2,3,4,5 seg�n el color) desde \lstinline|Source| (0,1...$k$) si son del suelo o de la f�brica $k$ respectivamente) hacia la fila \lstinline|Stair| y \lstinline|Chip| va a ser 0 o 1 en dependencia de si se tom� o no la ficha de jugador inicial. Si \lstinline|All_moves| $\neq \emptyset$  se va a seleccionar un movimiento en el siguiente orden:
\begin{itemize}
	\item[1.] 
	 \begin{itemize}
	 	\item Seleccionar de \lstinline|All_moves| aquellos movimientos que al realizarse completen exactamente una fila y con estos formar el conjunto \lstinline|Moves|.
	 	\item Si el jugador no lleva la delantera en la puntuaci�n tampoco van a pertenecer a \lstinline|Moves| aquellos movimientos que completar�an una fila en el muro en la fase 2 y, por tanto, finalizar�an la partida.
	 	\item Si \lstinline|Moves| $\neq \emptyset$:
	 	\begin{itemize}
	 		\item Ordenar \lstinline|Moves| de menor a mayor seg�n la cantidad de azulejos directamente conectados en l�nea recta en la fila y/o columna que tiene en el muro el color que se completar�a. 
	 		\item Seleccionar como movimiento a realizar el �ltimo elemento de \lstinline|Moves| y parar de buscar, con lo que se busca maximizar el n�mero de puntos a obtener en la fase de revestir el muro. 
	 	\end{itemize} 
	 \end{itemize}
 	\begin{lstlisting}
 	get_moves_overfill(ID, All_moves, K, Moves) :-
 		setof((Ady, Src, Clr, Amnt, St, Ch),
 		Free^(member((Src, Clr, Amnt, St, Ch), All_moves),
 	 	get_free_space(ID, St, Free), Amnt-Free=:=K,
 	  	get_adyacents(ID, St, Clr, Ady),
 	   	(is_not_game_move(ID, St, Clr);is_winning(ID))),
 		Moves).
 		
	strategy(ID, All_moves, Source, Color, Amount, Stair, Chip) :-
		get_moves_overfill(ID, All_moves, 0, Moves),
		last(Moves,
		(_, Source, Color, Amount, Stair, Chip)), !.
 	\end{lstlisting}
 	\item[2.] 
 	\begin{itemize}
 		\item Seleccionar de \lstinline|All_Moves| aquellos movimientos que al realizarse completen una fila y sobre exactamente un azulejo y con estos formar el conjunto \lstinline|Moves|.
 		\item Si el jugador no lleva la delantera en la puntuaci�n tampoco van a pertenecer a \lstinline|Moves| aquellos movimientos que completar�an una fila en el muro en la fase 2 y, por tanto, finalizar�an la partida. 
 	   \item Si \lstinline|Moves| $\neq \emptyset$:
 	\begin{itemize}
 		\item Ordenar \lstinline|Moves| de menor a mayor seg�n la cantidad de azulejos directamente conectados en l�nea recta en la fila y/o columna que tiene en el muro el color que se completar�a.
 		\item Seleccionar como movimiento a realizar el �ltimo elemento de \lstinline|Moves| y parar de buscar. 
 	 	\end{itemize}	
 	\end{itemize}
 \begin{lstlisting}
 	strategy(ID, All_moves, Source, Color, Amount, Stair, Chip) :-
 		get_moves_overfill(ID, All_moves, 1, Moves),
 		last(Moves,
 		(_, Source, Color, Amount, Stair, Chip)), !.
 \end{lstlisting}
 	\item[3.] 
 \begin{itemize}
 	\item Seleccionar de \lstinline|All_Moves| aquellos movimientos que al realizarse no alcancen a completar exactamente una fila y adem�s esta ya contenga al menos un azulejo, y con estos formar el conjunto \lstinline|Moves|.
 	\item Si \lstinline|Moves| $\neq \emptyset$:
 	\begin{itemize}
 		\item Ordenar \lstinline|Moves| de menor a mayor seg�n la cantidad de espacios vac�os que quedar�an en la fila al colocar los azulejos.
 		\item Seleccionar como movimiento a realizar el primer elemento de \lstinline|Moves| y parar de buscar.
 	\end{itemize}
  \end{itemize}
 \begin{lstlisting}
 	get_moves_incomplete_to_nonempty(ID, All_moves, Moves) :-
 		setof((Incomplete, Src, Clr, Amnt, St, Ch),
 		Free^(member((Src, Clr, Amnt, St, Ch), All_moves),
 		get_free_space(ID, St, Free), Incomplete is Free-Amnt,
 		Incomplete>0, stair(ID, St, 1, Clr)),
 		Moves).
 		
 	strategy(ID, All_moves, Source, Color, Amount, Stair, Chip) :-
 		get_moves_incomplete_to_nonempty(ID,All_moves,
 		[ (_, Source, Color, Amount, Stair, Chip)|_]), !.
 \end{lstlisting} 
	\item[4.] 
\begin{itemize}
	\item Seleccionar de \lstinline|All_Moves| aquellos movimientos que al realizarse no alcancen a completar exactamente una fila y con estos formar el conjunto \lstinline|Moves|.
	\item Si \lstinline|Moves| $\neq \emptyset$:
	\begin{itemize}
		\item Ordenar \lstinline|Moves| de menor a mayor seg�n la cantidad de espacios vac�os que quedar�an en la fila al colocar los azulejos.
		\item Seleccionar como movimiento a realizar el primer elemento de \lstinline|Moves| y parar de buscar.
	\end{itemize} 
\end{itemize}
\begin{lstlisting}
	get_moves_incomplete(ID, All_moves, Moves) :-
		setof((Incomplete, Src, Clr, Amnt, St, Ch),
		Free^(member((Src, Clr, Amnt, St, Ch), All_moves),
	 	get_free_space(ID, St, Free), Incomplete is Free-Amnt,
	  	Incomplete>0),Moves).
	  	
	strategy(ID, All_moves, Source, Color, Amount, Stair, Chip) :-
	  	get_moves_incomplete(ID,All_moves,	  	
	  	[ (_, Source, Color, Amount, Stair, Chip)|_]), !.
\end{lstlisting}
\item[5.] 
\begin{itemize}
	\item Seleccionar de \lstinline|All_Moves| aquellos movimientos que no provocan que se acabe el juego al llenar una fila del muro en la siguiente fase (LLegado a este punto estos movimientos se rechazaron anteriormente porque el jugador no contaba con la mayor puntuaci�n, por tanto ahora tampoco ser�a conveniente seleccionarlos, mientras que el resto de los movimientos tras haber sido descartados por las variantes anteriores solo pueden ser movimientos que completen una fila y sobre una cantidad mayor que 1 de azulejos que ir�an al suelo o que no se puedan colocar en  ninguna fila y vayan directamente al suelo) y con estos formar el conjunto \lstinline|Moves|.
	\item Si \lstinline|Moves| $\neq \emptyset$:
	\begin{itemize}
		\item Ordenar \lstinline|Moves| de menor a mayor seg�n la cantidad de azulejos que ser�an enviados al suelo.
		\item Seleccionar como movimiento a realizar el primer elemento de $F_5$ y parar de buscar.
	\end{itemize} 
\end{itemize}
\begin{lstlisting}
	strategy(ID, All_moves, Source, Color, Amount, Stair, Chip) :-
		setof((Extra, Src, Clr, Amnt, St, Ch),
		Free^(member((Src, Clr, Amnt, St, Ch), All_moves),
		get_free_space(ID, St, Free), Extra is Amnt-Free,
		is_not_game_move(ID, St, Clr)),	
		[ (_, Source, Color, Amount, Stair, Chip)|_]), !.
\end{lstlisting}
\item[6.] 
\begin{itemize}
	\item Alcanzar este punto significa que todos los movimientos en \lstinline|All_Moves| son aquellos que provocan el fin del juego al revestir el muro o que colocan las fichas directamente en el suelo

	\item Ordenar \lstinline|All_Moves| de menor a mayor seg�n la cantidad de azulejos que ser�an enviados al suelo.
	\item Seleccionar como movimiento a realizar el primer elemento de \lstinline|All_Moves|.
\end{itemize}
\begin{lstlisting}
	strategy(ID, All_moves, Source, Color, Amount, Stair, Chip) :-
		setof((Extra, Src, Clr, Amnt, St, Ch),
		Free^(member((Src, Clr, Amnt, St, Ch), All_moves),
	 	get_free_space(ID, St, Free), Extra is Amnt-Free),
		[ (_, Source, Color, Amount, Stair, Chip)|_]).
\end{lstlisting}
\end{itemize}


    




\end{document}
